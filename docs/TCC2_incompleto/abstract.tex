% !TeX root = document.tex
% !TeX encoding = UTF-8 Unicode

\begin{abstract}
    %Tendo em vista que [justificativa], pesquisa-se sobre [tema], a fim de [objetivo geral]. Para tanto, é necessário [objetivo específico 1], [objetivo específico 2] e [objetivo específico 3]. Realiza-se, então, uma pesquisa [metodologia científica]. Diante disso, verifica-se que [resultado 1], [resultado 2] e [resultado 3], o que impõe a constatação de que [conclusão]. Palavras-chave: [assunto]. [ponto de vista sobre o assunto]. [palavra de ligação entre o assunto e o ponto de vista].
    
    Tendo em vista que faz-se necessário investigar se o sistema robótico operando com um servo motor consegue equilibrar a bicicleta a uma velocidade diferente de zero, pesquisa-se sobre modelagem sistema e síntese de controladores, a fim de investigar a controlabilidade e sintetizar controladores que possam estabilizar o pendulo invertido em torno da posição de equilíbrio. Para tanto, é necessário modelar matematicamente os sistemas da planta de controle, Sintetizar controladores para os modelos obtidos e desenvolver um software para aplicar os controladores na planta de controle. Realiza-se, então, uma pesquisa de finalidade aplicada, objetivo exploratório, sob o método hipotético-dedutivo, com abordagem quantitativa, realizada com procedimentos experimentais. Nessa etapa do projeto, verifica-se que foi possível modelar o sistema MIMO como 3 subsistemas SISO e que a dinâmica do pendulo invertido varia conforme a velocidade.
\end{abstract}

Palavras-chave: Pêndulo invertido, modelagem de sistemas, síntese de controladores

\cleardoublepage{}

\begin{otherlanguage}{english}
	\begin{abstract}
        Considering that it is necessary to investigate whether the robotic system operating with a servo motor can balance the bicycle at a speed other than zero, research is carried out on system modeling and controller synthesis, in order to investigate the controllability and synthesize controllers that can stabilize the pendulum inverted around the equilibrium position. Therefore, it is necessary to mathematically model the control plant systems, Synthesize controllers for the obtained models and develop a software to apply the controllers in the control plant. It is then carried out a research of applied purpose, exploratory objective, under the hypothetical-deductive method, with a quantitative approach, carried out with experimental procedures. At this stage of the project, it is verified that it was possible to model the MIMO system as 3 SISO subsystems and that the inverted pendulum dynamics varies according to speed.
    \end{abstract}
    
    Keywords: Inverted pendulum, systems modeling, controller synthesis
\end{otherlanguage}
