\chapter{Conclusão}    \label{chp:conclusão}

    %Justificativa -> tema
    %Este trabalho de pesquisa foi iniciado diante da necessidade de se investigar se o sistema formado por um atuador de direção, do tipo servo-motor, acoplado a uma bicicleta pode ser estabilizado, considerando que este esteja em uma velocidade diferente de zero.

    %Objetivo Geral (se foi atendido)
    %Diante disso, o objetivo geral do trabalho é verificar se é possível sintetizar controladores que resultem em manter a inclinação do corpo em torno de uma posição de equilíbrio. Essa tarefa não pôde ser finalizada ainda. Tal objetivo deve ser atingido ao fim da segunda parte deste trabalho.
    
    %Cada um dos objetivos Específicos (se foi atingido)
    %O objetivo especifico inicial era modelar os sistemas da planta de controle. A planta de controle pode ser dividida em 3 subsistemas, os quais todos foram modeladas com as técnicas aprendidas no decorrer do curso. Logo, este objetivo foi atingido.
    
    %Já a respeito dos objetivos específicos de sintetizar controladores para os modelos obtidos e desenvolver um software para aplicar os controladores na planta de controle, não estavam previstos para serem atingidos no decorrer do desenvolvimento dessa parte do trabalho, estando previstos para serem realizados na segunda parte.
    
    %Relato das descobertas dos objetivos específicos
    %Foi descoberto que a dinâmica do pendulo invertido varia conforme a velocidade da planta é modificada, isso implica que controladores, projetados para equilibrar a planta em determinada velocidade, podem não ter exito  caso ocorra uma variação na velocidade. Uma abordagem para resolver esse impasse é controlar a velocidade tangencial, de maneira que essa variável permaneça próxima do valor pré determinado na linearização do modelo.
    
    %Hipótese (Qual foi? Confirmada ou refutada)
    %Partiu-se da hipótese que é possível equilibrar uma bicicleta sem deslocamento de massa, apenas alterando o eixo de direção com um atuador que cancela o sistema de controle mecânico. Essa hipótese não pode ser confirmada ou refutada ainda, devido ao estágio inicial do trabalho, sendo previsto para a segunda parte do mesmo.
    
    %Limitações: Dificuldades encontradas - Limitações da pesquisa
    %A principal dificuldade encontrada durante a realização do trabalho foi o tempo reduzido para sua execução. Foi necessário escolher qual objetivo seria atingido inicialmente, deixando o restante para trabalhos futuros.
    
    %Recomendações: Consequência das dificuldades, recomendar continuidade
    %TCC2
